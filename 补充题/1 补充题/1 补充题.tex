%!TEX TX-program = xelatex
\documentclass[8pt]{article}

\usepackage[UTF8]{ctex}
\usepackage{graphicx}
\usepackage{enumerate}
\usepackage{geometry}
\usepackage{amsmath}
\usepackage{amssymb}
\usepackage{amsfonts}

\author{高一(6)班\ 邵亦成\ 26号}
\title{1 补充题}
\date{2021年10月11日}

\geometry{a4paper, scale=0.8}

\begin{document}

	\maketitle

	\begin{enumerate}[(1)]
		\item
			已知$x>0, a>0$.求$f(x)=x+\frac{a}{x}$的$f_{\min}(x)$并求出对应的值.

			~\\
			$\because x>0, a>0 \therefore \frac{a}{x}>0.$

			由基本不等式,有

			$$f(x)=x+\frac{a}{x}\geq2\sqrt{x\cdot\frac{a}{x}}=2\sqrt{a}.$$

			取等当且仅当$x=\frac{a}{x}$即$x=\pm\sqrt{a}$(舍负).

			于是有$f_{\min}(x)=f(\sqrt{a})=2\sqrt{a}.$

		\item
			已知$a>0$.求$f(x)=x+\frac{a}{x}$的值域.

			~\\
			易得$x\neq0$.

			分两类讨论.

			\begin{enumerate}[$1^\circ$]
				\item
					$x>0,$

					由(1)有$f_{\min}(x)=f(\sqrt{a})=2\sqrt{a}.$

				\item
					$x<0,$

					$\because x<0, a>0 \therefore \frac{a}{x}<0.$

					由基本不等式,有

					$$f(x)=x+\frac{a}{x}\leq-2\sqrt{x\cdot\frac{a}{x}}=-2\sqrt{a}.$$

					取等当且仅当$x=\frac{a}{x}$即$x=\pm\sqrt{a}$(舍正).

					于是有$f_{\max}(x)=f(-\sqrt{a})=-2\sqrt{a}.$

			\end{enumerate}

			综上,$f(x)$的值域为$\left(-\infty, -2\sqrt{a}\right]\cup\left[2\sqrt{a}, +\infty\right)$.

		\item
			已知$x>1$.求$f(x)=x+\frac{1}{x-1}$的最小值.

			~\\
			$\because x>1 \therefore x>1>0, \frac{1}{x-1}>0.$

			由基本不等式,有

			$$f(x)-1=x+\frac{a}{x-1}-1=(x-1)+\frac{1}{x-1}\geq2\sqrt{(x-1)\cdot\frac{1}{x-1}}=2.$$

			取等当且仅当$x-1=\frac{1}{x-1}$即$x-1=\pm1$即$x=\pm1+1=2$或$0$(舍).

			于是有$f_{\min}(x)=f(2)=3.$

		\item
			求$f(x)=\frac{x^2+x+1}{x+1}$的值域.

			~\\
			$f(x)=x+\frac{1}{x+1}.$

			易得$x\neq-1$.

			分两类讨论.

			\begin{enumerate}[$1^\circ$]
				\item
					$x>-1,$

					$\because x>-1 \therefore x+1>0, \frac{1}{x+1}>0.$

					由基本不等式,有

					$$f(x)+1=x+\frac{1}{x+1}+1=(x+1)+\frac{1}{x+1}\geq2\sqrt{(x+1)\cdot\frac{1}{x+1}}=2.$$

					取等当且仅当$x+1=\frac{1}{x+1}$即$x+1=\pm1$即$x=0$或$-2$(舍).

					于是有$f_{\min}(x)=f(0)=1.$

				\item
					$x<-1,$

					$\because x<-1 \therefore x+1<0, \frac{1}{x+1}<0.$

					由基本不等式,有

					$$f(x)+1=x+\frac{1}{x+1}+1=(x+1)+\frac{1}{x+1}\leq-2\sqrt{(x+1)\cdot\frac{1}{x+1}}=-2.$$

					取等当且仅当$x+1=\frac{1}{x+1}$即$x+1=\pm1$即$x=-2$或$0$(舍).

					于是有$f_{\max}(x)=f(-2)=-3.$

			\end{enumerate}

			综上,$f(x)$的值域为$\left(-\infty, -3\right]\cup\left[1,+\infty\right)$.

		\item
			求$f(x)=\frac{2x}{x^2+x+1}$的值域.

			~\\
			分两类讨论.

			\begin{enumerate}[$1^\circ$]
				\item
					$x=0,$

					$f(0)=0$

				\item
					$x\neq 0,$

					$f(x)=\frac{2}{x+1+\frac{1}{x}}.$

					由(2)有$g(x)=x+\frac{1}{x}$的值域为$\left(-\infty,-2\right]\cup\left[2,+\infty\right).$

					$\therefore g(x)+1$的值域为$\left(-\infty,-1\right]\cup\left[3,+\infty\right).$

					$\therefore \frac{1}{g(x)+1}$的值域为$\left[-1,0\right)\cup\left(0,\frac{1}{3}\right].$

					$\therefore \frac{2}{g(x)+1}=f(x)$的值域为$\left[-2,0\right)\cup\left(0,\frac{2}{3}\right].$

			\end{enumerate}

			综上,$f(x)$的值域为$\left[-2,\frac{2}{3}\right].$

	\end{enumerate}

\end{document}
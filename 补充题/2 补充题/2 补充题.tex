%!TEX TX-program = xelatex
\documentclass[8pt]{article}

\usepackage{ctex}
\usepackage{graphicx}
\usepackage{enumitem}
\usepackage{geometry}
\usepackage{amsmath}
\usepackage{amssymb}
\usepackage{amsfonts}
\usepackage{tikz}
\usepackage{extarrows}
\usetikzlibrary{positioning}
\usetikzlibrary{svg.path}
\usepackage{xcolor}
\usepackage{soul}

\graphicspath{ {./images/} }

\author{高一(6)班\ 邵亦成\ 26号}
\title{2 补充题}
\date{2021年10月31日}

\geometry{a4paper, scale=0.8}

\begin{document}

	\maketitle

	函数$f(x)$对于任意实数$f(x)$满足条件:

	$$f(x+2)=\frac{1}{f(x)},$$

	且有$f(0)=1, f(1)=-5.$

	\begin{enumerate}[label=(\arabic*)]
		\item 计算$f(2), f(4), f(6), f(8)$的值, 并猜想$f(2n)$的结果, 其中$n \in \mathbf{N}.$
		\item 计算$f(3), f(5), f(7), f(9)$的值, 并猜想$f(2n-1)$的结果, 其中$n \in \mathbf{N}.$
		\item 试证明(1), (2)所猜想的结果.
	\end{enumerate}
	~\\

	\begin{enumerate}[label=(\arabic*)]
		\item 计算$f(2), f(4), f(6), f(8)$的值, 并猜想$f(2n)$的结果, 其中$n \in \mathbf{N}.$

			$$f(2)=\frac{1}{f(2-2)}=\frac{1}{f(0)}=\frac{1}{1}=1,$$

			$$f(4)=\frac{1}{f(4-2)}=\frac{1}{f(2)}=\frac{1}{1}=1,$$

			$$f(6)=\frac{1}{f(6-2)}=\frac{1}{f(4)}=\frac{1}{1}=1,$$

			$$f(8)=\frac{1}{f(8-2)}=\frac{1}{f(6)}=\frac{1}{1}=1,$$

			猜想:

			$$f(2n)=1, n \in \mathbf{N}. \eqno{(1)}$$

		\item 计算$f(3), f(5), f(7), f(9)$的值, 并猜想$f(2n-1)$的结果, 其中$n \in \mathbf{N}.$

			$$f(3)=\frac{1}{f(3-2)}=\frac{1}{f(1)}=\frac{1}{-5}=-\frac{1}{5},$$

			$$f(5)=\frac{1}{f(5-2)}=\frac{1}{f(3)}=\frac{1}{-\frac{1}{5}}=-5,$$

			$$f(7)=\frac{1}{f(7-2)}=\frac{1}{f(5)}=\frac{1}{-5}=-\frac{1}{5},$$

			$$f(9)=\frac{1}{f(9-2)}=\frac{1}{f(7)}=\frac{1}{-\frac{1}{5}}=-5,$$

			猜想:

			$$f(2n-1)=
			\left\{
			\begin{array}{rl}
			-5,& n\text{是奇数},\\
			\displaystyle -\frac{1}{5}, &n\text{是偶数}.\\
			\end{array}
			\right.
			\eqno{(2)}
			$$

		\item 试证明 (1), (2) 所猜想的结果.

			\textbf{下证 (1) 式成立:}

				考虑$n=0$, 有

				$$f(2\times0)=f(0)=1$$

				符合 (1) 式.

				假设 (1) 式对$n=k$成立, 下证其对$n=k+1$成立.

				$$f[2(k+1)]=f(2k+2)=\frac{1}{f(2k+2-2)}=\frac{1}{f(2k)}=\frac{1}{1}=1$$

				成立.

				由第一数学归纳法, (1) 式成立.

			\textbf{下证 (2) 式成立:}

				考虑$n=0$, 有

				$$f(1)=\frac{1}{f(1-2)}=\frac{1}{f(-1)}=\frac{1}{f(2\times 0-1)}=-5 \Rightarrow f(2\times0-1)=f(-1)=-\frac{1}{5}$$

				符合 (2) 式.

				考虑$n=1$, 有

				$$f(2\times 1-1)=f(1)=-5$$

				符合 (2) 式.

				假设 (2) 式对$n=2k$成立, 下证其对$n=2k+2$成立.

				$$f[2(2k+2)-1]=f(4k+3)=\frac{1}{f(4k+3-2)}=\frac{1}{f(4k+1)}=\frac{1}{\frac{1}{f(4k+1-2)}}=\frac{1}{\frac{1}{f(4k-1)}}=f(4k-1)=f[2(2k)-1]=-\frac{1}{5}$$

				成立.

				假设 (2) 式对$n=2k+1$成立, 下证其对$n=2k+3$成立.

				$$f[2(2k+3)-1]=f(4k+5)=\frac{1}{f(4k+5-2)}=\frac{1}{f(4k+3)}=\frac{1}{\frac{1}{f(4k+3-2)}}=\frac{1}{\frac{1}{f(4k+1)}}=f(4k+1)=f[2(2k+1)-1]=-5$$

				成立.

				由第一数学归纳法, (2) 式成立.

	\end{enumerate}

\end{document}